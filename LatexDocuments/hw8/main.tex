\documentclass{article}

% Language setting
% Replace `english' with e.g. `spanish' to change the document language
% \usepackage[english]{babel}
\usepackage[UTF8,heading = true]{ctex}

% Set page size and margins
% Replace `letterpaper' with `a4paper' for UK/EU standard size
\usepackage[a4paper,top=2cm,bottom=2cm,left=3cm,right=3cm,marginparwidth=1.75cm]{geometry}

\usepackage{float}
\usepackage{titlesec}
\usepackage{lipsum}
\usepackage{zhnumber}
% 表格
\usepackage{diagbox}
\usepackage{makecell}
% Useful packages
\usepackage{amsmath}
\usepackage{graphicx}
\usepackage[colorlinks=true, allcolors=blue]{hyperref}
\usepackage{fancyhdr} % 页眉页脚

% 画图
\usepackage{tikz}
\usepackage{pgfplots}  
\pgfplotsset{width=10cm,compat=1.13}

\fancyhf{} % 清除所有页眉页脚
\pagestyle{fancy}
\fancypagestyle{plain}{%改变首页
  \fancyhf{}
  \cfoot{\thepage}
  % \fancyhead[C]{大数据机器学习}
  \renewcommand{\headrulewidth}{0mm} 
}

\fancyfoot[CO, CE]{\thepage}
\renewcommand{\headrulewidth}{0mm} 
% 设置序号格式
% \renewcommand{\thesection}{\arabic{section}.}
\renewcommand{\thesection}{\zhnum{section}、 }
% \renewcommand{\thesubsection}{\zhnum{subsection}、}
\renewcommand{\thesubsection}{\Roman{subsection}.}
\renewcommand{\thesubsubsection}{\arabic{subsubsection})}
% 设置标题的字体
\titleformat{\section}
  {\normalfont}{\thesection}{1em}{}
\titleformat{\subsection}
  {\normalfont}{\thesubsection}{1em}{}

% 代码和伪代码支持
\RequirePackage{minted}
\RequirePackage[ruled,linesnumbered]{algorithm2e}
\usepackage{listings}
\lstset{
    basicstyle          =   \sffamily,          % 基本代码风格
    keywordstyle        =   \bfseries,          % 关键字风格
    commentstyle        =   \rmfamily\itshape,  % 注释的风格,斜体
    stringstyle         =   \ttfamily,  % 字符串风格
    flexiblecolumns,                % 别问为什么,加上这个
    % numbers             =   left,   % 行号的位置在左边
    showspaces          =   false,  % 是否显示空格,显示了有点乱,所以不现实了
    % numberstyle         =   \zihao{-5}\ttfamily,    % 行号的样式,小五号,tt等宽字体
    showstringspaces    =   false,
    % captionpos          =   t,      % 这段代码的名字所呈现的位置,t指的是top上面
    % frame               =   lrtb,   % 显示边框
}
\lstdefinestyle{Python}{
    language        =   Python, % 语言选Python
    basicstyle      =   \zihao{-5}\ttfamily,
    numberstyle     =   \zihao{-5}\ttfamily,
    keywordstyle    =   \color{blue},
    keywordstyle    =   [2] \color{teal},
    stringstyle     =   \color{magenta},
    commentstyle    =   \color{red}\ttfamily,
    breaklines      =   true,   % 自动换行,建议不要写太长的行
    columns         =   fixed,  % 如果不加这一句,字间距就不固定,很丑,必须加
    basewidth       =   0.5em,
}

% 数学支持
\RequirePackage{amssymb}
\RequirePackage{amsfonts}
\RequirePackage{amsmath}
\RequirePackage{esint}
\RequirePackage{gensymb}
\RequirePackage{mathtools}

\title{大数据机器学习第八次理论作业}
% \author{}
\begin{document}
\maketitle

\section{给定盒子和球组成的隐马尔可夫模型$\lambda=(A,B,\pi)$,其中,$$
A = \left[\begin{matrix}0.5 &0.2 &0.3\\0.3 &0.5 &0.2\\0.2 &0.3 &0.5\end{matrix}\right],B = \left[\begin{matrix}0.5 &0.5\\0.4 &0.6\\0.7 &0.3\end{matrix}\right],\pi = (0.2,0.4,0.4)^T
$$设$T=4$,$O=(\text{红},\text{白},\text{红},\text{白})$试用维特比算法求最优路径$I^*=(i_1^*,i_2^*,i_3^*,i_4^*)$。}

\subsection{问题分析与相关知识点}
\subsubsection{隐马尔可夫模型}
隐马尔可夫模型是关于时序的概率模型。描述由一个隐藏的马尔可夫链随机生成不可观测的状态随机序列,再由各状态生成一个观测而产生观测随机序列的过程,序列的每个位置也可看作是一个时刻。
隐马尔可夫模型的组成元素:
\begin{itemize}
    \item $Q$:所有可能状态的集合,$Q=\{q_1,q_2,\cdots,q_N\}$
    \item $V$:所有可能观测的集合,$V=\{v_1,v_2,\cdots,v_N\}$
    \item $I$:长度为T的状态序列,$I=\{i_1,i_2,\cdots,i_T\}$
    \item $O$:对应的长度T的观测序列,$O=\{o_1,o_2,\cdots,o_T\}$
    \item $\pi$:初始状态概率向量
    \item $A$:状态转移概率矩阵
    \[A=[a_{ij}]_{N \times N}\]
    \[a_{ij}=P(i_{t+1}=q_j|i_t=q_i),\quad i=1,2,\cdots,N;\quad j = 1,2,\cdots,N\]
    \item $B$:观测概率矩阵
    \[B=[b_j(k)]_{N \times M}\]
    \[b_j(k)=P(o_t=v_k|i_t=q_i),\quad k=1,2,\cdots,M;\quad j = 1,2,\cdots,N\]
\end{itemize}
三要素为:
\[\lambda=(A,B,\pi)\]
两个基本假设为:
\begin{enumerate}
    \item 齐次马尔可夫性假设,隐马尔可夫链t的状态只和t-1的状态有关:
    \[P(i_t|i_{t-1},o_{t-1},\cdots,i_1,o_1)=P(i_t|i_t-1),\quad t=1,2,\cdots,T
    \]
    \item 观测独立性假设,观测只和当前时刻状态有关:
    \[P(o_t|i_T,o_T,i_{T-1},o_{T-1},\cdots,i_{t+1},o_{t+1},i_t,i_{t-1},o_{t-1},\cdots,i_1,o_1)=P(o_t|i_t)\]
\end{enumerate}
\subsubsection{隐马尔可夫模型的预测(解码)问题}
隐马尔可夫模型的问题可分为计算$P(O|\lambda)$的概率计算问题、估计使$P=(O|\lambda)$最大的$\lambda=(A,B,\pi)$的学习问题以及求使$P=(I|O)$最大的状态序列$I = (i_1,i_2,\cdots,i_T)$的预测问题。

马尔科夫模型的预测的两种算法为:近似算法与维特比算法(Viterbi algorithm)。近似算法计算简单,但不能保证预测的状态序列是整体最有可能的状态序列。

维特比算法使用动态规划解马尔可夫模型,用动态规划求概率最大路径。这时一条路径对应着一个状态序列。

根据动态规划的原理,最优路径的特性为:如果最优路径在时刻t通过结点$i_t^*$,那么这一路径的从结点$i_t^*$到终点$i_T^*$的部分路径,对从$i_t^*$到终点$i_T^*$的所有可能的部分路径来说必须为最优。根据这一原理,我们从$t=1$时刻开始,递推地计算在时刻$t$状态$i$的各条部分路径的最大概率,直到得到时刻$T$状态为$i$的各条路径的最大概率$P*$和最优路径终结点$i_T^*$。之后为了找出最优路径的结点,从终点$i_T^*$开始逐步向前得到结点$i_{T-1}^*,\cdots,i_1^*$,得到最优路径$I^* = (i_1^*,i_2^*,\cdots,i_T^*)$。

我们导入两个变量$\delta$和$\Psi$。
\begin{itemize}
    \item 定义在时刻$t$状态为$i$的所有单个路径$(i_1,i_2,\cdots,i_t)$的概率最大值为$\delta_t(i)$
$$\delta_t(i)=\mathop{max}\limits_{i_1.i_2,\cdots,i_{t-1}} P(i_t=i,i_{t-1},\cdots,i_1,o_t,\cdots,o_1|\lambda),\quad i=1,2,\cdots,N
$$
由定义可得变量$\delta$的递推公式:
$$
\begin{align}
    \delta_{t+1}(i)&=\mathop{max}\limits_{i_1.i_2,\cdots,i_{t-1}} P(i_t=i,i_{t-1},\cdots,i_1,o_t,\cdots,o_1|\lambda)\\
    &=\mathop{max}\limits_{1\leq j \leq N} [\delta_t(j)a_{ji}]b_i(o_{t+1}),\quad i=1,2,\cdots,N;t=1,2,\cdots,T-1
\end{align}
$$
    \item 定义在时刻时刻$t$状态为$i$的所有单个路径$(i_1,i_2,\cdots,i_{t-1},i)$中概率最大路径的第$t-1$个结点为$\Psi_t(i)$:
    $$
    \Psi_t(i)=\arg \max_{1 \leq j \leq N}[\delta_{t-1}(j)a_{ji}],\quad i=1,2,\cdots,N
     $$
\end{itemize}
\subsubsection{维特比算法}
输入:模型$\lambda=(A,B,\pi)$和观测$O=(o_1,o_2,\cdots,o_T)$

输出:最优路径$I^* = (i_1^*,i_2^*,\cdots,i_T^*)$
过程:
\begin{enumerate}
    \item 初始化:
    $$
    \delta_1(i)=\pi_ib_i(o_1),\quad i=1,2,\cdots,N
    $$
    $$
    \Psi_1(i)=0,\quad i=1,2,\cdots,N
    $$
    \item 递推,对$t=2,3,\cdots,T$:
    $$
    \delta_t(i)=\max_{1\leq j \leq N} [\delta_{t-1}(j)a_{ji}]b_i(o_t),\quad i=1,2,\cdots,N
    $$
    $$
    \Psi_t(i)=\arg \max_{1 \leq j \leq N}[\delta_{t-1}(j)a_{ji}],\quad i=1,2,\cdots,N
    $$
    \item 终止:
    $$
    P*=\max_{1 \leq j \leq N} \delta_T(i)
    $$
    $$
    i_T^* = \arg \max_{1 \leq j \leq N}[\delta_T(i)]
    $$
    \item 最优路径回溯,对$t=T-1,T-2,\cdots,1$:
    $$
    i_t^*=\Psi_{t+1}(i_{t+1}^*)
    $$
    求得最优路径$I^* = (i_1^*,i_2^*,\cdots,i_T^*)$。
    
\end{enumerate}

\subsection{求解最优路径}
\begin{enumerate}
    \item 初始化,在$t=1$时,对每个状态$i$,$i=1,2,3$,求观测状态为$i$观测$o_1$为红的概率。记概率为$delta_1(i)$,则:
    $$
\delta_1(i)=\pi_ib_i(o_1)=\pi_ib_i(\text{红}),\quad i=1,2,3$$
    代入数据:
    $$
     \delta_1(1)=0.10,\quad \delta_1(2)=0.16,\quad \delta_1(3)=0.28
    $$
    记$\Psi_1(i)=0,i=1,2,3$

    \item 在$t=2$时,对每个状态$i$,$i=1,2,3$,求$t=1$时状态为$j$观测为红并在$t=2$时状态为$i$观测$o_2$为白的路径的最大概率,记此最大概率为$\delta_2(i)$,则
    $$
    \delta_2(i)=\max\limits_{1\leq j \leq 3} [\delta_1(j)a_{ji}]b_i(o_2)
    $$
    同时,对每个状态$i$,记录概率最大路径的前一个状态$j$:
    $$
    \Psi_2(i)=\arg \max_{1 \leq j \leq 3}[\delta_1(j)a_{ji}],\quad i=1,2,3
    $$
    计算:
    $$
    \begin{align}
        \delta_2(1)
        &=\max\{0.10\times 0.5,0.16\times 0.3,0.28\times 0.2\}\times 0.5\\
        &=0.028\\
        \Psi_2(1)&=3\\
        \delta_2(2)&=\max\{0.10\times 0.2,0.16\times 0.5,0.28\times 0.3\}\times 0.6\\&=0.0504\\
        \Psi_2(2)&=3\\
        \delta_2(3)&=\max\limits_{1\leq j \leq 3} [\delta_1(j)a_{j3}]b_3(o_2)\\&=\max\{0.10\times 0.3,0.16\times 0.2,0.28\times 0.5\}\times 0.3\\&=0.042\\
        \Psi_2(3)&=3
    \end{align}
    $$
    同理,在$t=3$时:
    $$
    \begin{align}
    \delta_3(i)&=\max\limits_{1\leq j \leq 3} [\delta_2(j)a_{ji}]b_i(o_3)\\
    \Psi_3(i)&=\arg \max_{1 \leq j \leq 3}[\delta_2(j)a_{ji}]\\
        \delta_3(1)&=\max\{0.028\times 0.5,0.0504\times 0.3,0.042\times 0.2\}\times 0.5\\
        &=0.00756\\
        \Psi_3(1)&=2\\
        \delta_3(2)&=\max\{0.028\times 0.2,0.0504\times 0.5,0.042\times 0.3\}\times 0.4\\&=0.01008\\
        \Psi_3(2)&=2\\
        \delta_3(3)&=\max\{0.028\times 0.3,0.0504\times 0.2,0.042\times 0.5\}\times 0.7\\&=0.0147\\
        \Psi_3(3)&=3
    \end{align}
    $$
    同理,在$t=4$时:
    $$
    \begin{align}
    \delta_4(i)&=\max\limits_{1\leq j \leq 3} [\delta_3(j)a_{ji}]b_i(o_4)\\
    \Psi_4(i)&=\arg \max_{1 \leq j \leq 3}[\delta_3(j)a_{ji}]\\
        \delta_4(1)&=\max\{0.00756\times 0.5,0.01008\times 0.3,0.0147\times 0.2\}\times 0.5\\
        &=0.00189\\
        \Psi_4(1)&=1\\
        \delta_4(2)&=\max\{0.00756\times 0.2,0.01008\times 0.5,0.0147\times 0.3\}\times 0.6\\&=0.003024\\
        \Psi_4(2)&=2\\
        \delta_4(3)&=\max\{0.00756\times 0.3,0.01008\times 0.2,0.0147\times 0.5\}\times 0.3\\&=0.002205\\
        \Psi_4(3)&=3
    \end{align}
    $$
    \item 以$P^*$表示最优路径概率,则
    $$
    P^*=\max\limits_{1\leq j \leq 3} [\delta_4(i)]=0.003024
    $$
    最优路径的终点为$i_4^*$:
    $$
    i_4^*==\arg \max_i[\delta_4(i)]=2
    $$
    \item 由最优路径的终点$i_4^*$,逆向找到$i_3^*,i_2^*,i_1^*$:
    $$
    t=3\text{时,}i_3^*=\Psi_4(i_4^*)=\Psi_4(2)=2
    $$
    $$
    t=2\text{时,}i_2^*=\Psi_3(i_3^*)=\Psi_3(2)=2
    $$
    $$
    t=1\text{时,}i_1^*=\Psi_2(i_2^*)=\Psi_2(2)=3
    $$
    于是求得最优路径,即最优状态序列$I^*=(i_1^*,i_2^*,i_3^*,i_4^*)=(3,2,2,2)$。
    \begin{figure}[H]
    \centering
    \begin{tikzpicture}
    \begin{axis}[
    xlabel= \text{时间},
    ylabel = \text{状态},
    legend cell align = left,
    % tick align = inside,
    % grid=major,
    axis y line=left,
	axis x line=bottom,
    xmin=0,xmax=5,
    ymin=0,ymax=4,
    xtick={1,2,3,4},
    ytick={1,2,3}
    ]
    \addplot[purple]
     plot coordinates{
    (1,3)
    (2,2)
    (3,2)
    (4,2)
    };
    \addplot[black]
     plot coordinates{
    (1,3)
    (2,3)
    (3,3)
    (4,3)
    };
    \addplot[black]
     plot coordinates{
    (2,2)
    (3,1)
    (4,1)
    };
    \addplot[black]
     plot coordinates{
    (1,3)
    (2,1)
    };
    \node at (1,3) [label = 0.28,shape=circle,scale=0.5,fill=purple] ;
    \node at (1,2) [label = 0.16,shape=circle,scale=0.5,fill=black] ;
    \node at (1,1) [label = 0.10,shape=circle,scale=0.5,fill=black] ;
    \node at (2,3) [label = 0.042,shape=circle,scale=0.5,fill=black] ;
    \node at (2,2) [label = 0.0504,shape=circle,scale=0.5,fill=purple] ;
    \node at (2,1) [label = 0.028,shape=circle,scale=0.5,fill=black] ;
    \node at (3,3) [label = 0.0147,shape=circle,scale=0.5,fill=black] ;
    \node at (3,2) [label = 0.01008,shape=circle,scale=0.5,fill=purple] ;
    \node at (3,1) [label = 0.00765,shape=circle,scale=0.5,fill=black] ;
    \node at (4,3) [label = 0.002205,shape=circle,scale=0.5,fill=black] ;
    \node at (4,2) [label = 0.003024,shape=circle,scale=0.5,fill=purple] ;
    \node at (4,1) [label = 0.00189,shape=circle,scale=0.5,fill=black] ;
    \end{axis}
    \end{tikzpicture}
    \end{figure}
\end{enumerate}
\subsection{代码验证}

\paragraph{代码实现}
\lstinputlisting[
    style       =   Python,
    % caption     =   {\bf decisionstump.py},
    label       =   {hmm-viterbi.py}
]{hmm-viterbi.py}
\paragraph{输出结果}
\lstinputlisting[
    style       =   Python,
    % caption     =   {\bf decisionstump.py},
    label       =   {hmm_res_1.py}
]{hmm_res_1.py}

\subsection{总结}
代码计算的结果与之前的结果一致,最优状态序列为$I^*=(3,2,2,2)$。
\section{试用前向概率和后向概率推导:$$P(O|\lambda) = \sum_{i=1}^N\sum_{j=1}^N\alpha_t(i)a_{ij}b_j(o_{t+1})\beta_{t+1}(j),\quad t=1,2,\dots,T-1$$}
\subsection{问题分析与相关知识点}
\subsubsection{前向概率}
给定马尔科夫模型$\lambda$,定义到时刻$t$部分观测序列为:$o_1,o_2,\cdots,o_t$,且状态为$q_i$的概率为前向概率,记为:
$$
\alpha_t(i)=P(o_1,o_2,\cdots,o_t,i_t=q_i|\lambda)
$$
可以递推地求前向概率$\alpha_t(i)$以及观测序列概率$P(O|\lambda)$。

\subsubsection{观测序列概率的前向算法}
输入:隐马尔可夫模型$\lambda$,观测序列$O$

输出:观测序列概率$P(O|\lambda)$
\begin{enumerate}
    \item 初始化前向概率:
    $$
    \alpha_1(i)=\pi_ib_i(o_1),\quad i=1,2,\cdots,N
    $$
    \item 递推,对$t=1,2,\cdots,T-1$,根据马尔可夫模型的两个基本假设,我们基于前一时刻的各状态的前向概率,乘以对应的状态转移概率,以及观测概率,得到时刻$t+1$的各个状态$q_i$的前向概率。
    $$
    \alpha_{t+1}(i)=\left[\sum_{j=1}^N\alpha_t(j)a_{ji}\right]b_i(o_{t+1}),\quad i=1,2,\cdots,N
    $$
    \item 终止,将时刻$T$各状态对应的概率相加,得到观测序列概率:
    $$
    P(O|\lambda)=\sum_{i=1}^N\alpha_T(i)
    $$
\end{enumerate}
\begin{figure}[H]
    \centering
    \includegraphics[width=0.5\linewidth]{hmm-forward.png}
    % \caption{前向概率递推公式}
    \label{fig:enter-label}
\end{figure}
\subsubsection{后向概率}
给定马尔可夫模型$\lambda$,定义到时刻$t$状态为$q_i$条件下,从$t+1$到$T$的部分观测序列为$o_{t+1},o_{t+2},\cdots,o_T$的概率为后向概率,记作
$$
\beta_t(i)=P(o_{t+1},o_{t+2},\cdots,o_T|i_t=q_i,\lambda)
$$可以用递推的方法求后向概率$\beta_t(i)$以及观测序列概率$P(O|\lambda)$
\subsubsection{观测序列概率的后向算法}
输入:隐马尔可夫模型$\lambda$,观测序列$O$

输出:观测序列概率$P(O|\lambda)$
\begin{enumerate}
    \item 初始化后向概率:
    $$
    \beta_T(i)=1,\quad i=1,2,\cdots,N$$
    \item 递推,对$t=T-1,T-2,\cdots,1$,根据后一时刻的各状态的后向概率、对应的状态转移概率以及观测概率计算时刻$t$的各个状态$q_i$的后向概率。
    $$
    \beta_t(i)=\sum_{j=1}^N a_{ij}b_j(o_{t+1})\beta_{t+1}(j),\quad i=1,2,\cdots,N$$
    \item 终止:
    $$
    P(O|\lambda)=\sum_{i=1}^N\pi_ib_i(o_1)\beta_1(i)$$
\end{enumerate}
\begin{figure}[H]
    \centering
    \includegraphics[width=0.5\linewidth]{hmm-after.png}
    \label{fig:enter-label}
\end{figure}
\subsection{问题推导}
观测序列概率$P(O|\lambda)$即观测序列$O$在模型参数$\lambda=(\pi,A,B)$下出现的条件概率。$I$是可能出现的隐藏的状态序列,对任意一个状态序列I,有:
$$
P(I|\lambda) = P(i_1,i_2,\cdots,i_T|\lambda)=\pi_{i_1}a_{i_1i_2}\cdots a_{i_{T-1}i_T}
$$
且:
$$P(O|I,\lambda) = P(o_1,o_2,\cdots,o_T|i_1,i_2,\cdots,i_T,\lambda)=b_{i_1}(o_1)b_{i_2}(o_2)\cdots b_{i_T}(o_T)
$$
$$
P(O,I|\lambda) = P(O|I,\lambda)P(I|\lambda)=\pi_{i_1}b_{i_1}(o_1)a_{i_1i_2}b_{i_2}(o_2)\cdots a_{i_{T-1}i_T}b_{i_T}(o_T)
$$
根据联合概率可求边缘概率:
$$
\begin{align}
P(O|\lambda) &= \sum_I P(O,I|\lambda)\\
&=\sum_{i_1,i_2,\cdots,i_T}\pi_{i_1}b_{i_1}(o_1)a_{i_1i_2}b_{i_2}(o_2)\cdots a_{i_{T-1}i_T}b_{i_T}(o_T)
\end{align}
$$
直接计算上式时间复杂度$O(TN^T)$过高,我们根据前向概率和后向概率的定义推导:
$$
\begin{align}
P(O|\lambda) &=\sum_{i_t,i_{t+1}} P(O,i
_t,i_{i+1}|\lambda)\\
&=\sum_{i=1}^N \sum_{j=1}^N P(O,i
_t=q_i,i_{t+1}=q_j|\lambda)\\
&=\sum_{i=1}^N \sum_{j=1}^N P(o_1,o_2,\cdots,o_{t-1},\cdots,o_T,i_t=q_i,i_{t+1}=q_j|\lambda)\\
&=\sum_{i=1}^N \sum_{j=1}^N P(o_1,o_2,\cdots,o_{t-1},\cdots,o_t,i_t=q_i|\lambda)P(i_{t+1}=q_j,o_{t+1},o_{t+2},\cdots,\cdots,o_T|\lambda)\\
&=\sum_{i=1}^N \sum_{j=1}^N \alpha_t(i) P(i_{t+1}=q_j|\lambda)  P(o_{t+1},o_{t+2},\cdots,\cdots,o_T|i_{t+1}=q_j,\lambda)\\
&=\sum_{i=1}^N \sum_{j=1}^N \alpha_t(i) P(i_{t+1}=q_j|\lambda) P(o_{t+1}|i_{t+1}=q_j,\lambda) P(o_{t+2},\cdots,\cdots,o_T|i_{t+1}=q_j,\lambda)\\
&=\sum_{i=1}^N \sum_{j=1}^N \alpha_t(i)a_{ij}b_j(o_{t+1})\beta_{t+1}(j)
\end{align}
$$
推导完毕。
\subsection{总结}
使用前向概率加上后向概率可以计算整体的观测序列概率,且相比于直接计算法,可以减少时间复杂度。
\end{document}