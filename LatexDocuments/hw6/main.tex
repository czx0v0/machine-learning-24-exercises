\documentclass{article}

% Language setting
% Replace `english' with e.g. `spanish' to change the document language
% \usepackage[english]{babel}
\usepackage[UTF8,heading = true]{ctex}

% Set page size and margins
% Replace `letterpaper' with `a4paper' for UK/EU standard size
\usepackage[a4paper,top=2cm,bottom=2cm,left=3cm,right=3cm,marginparwidth=1.75cm]{geometry}

\usepackage{float}
\usepackage{titlesec}
\usepackage{lipsum}
\usepackage{zhnumber}
% 表格
\usepackage{diagbox}
\usepackage{makecell}
% Useful packages
\usepackage{amsmath}
\usepackage{graphicx}
\usepackage[colorlinks=true, allcolors=blue]{hyperref}
\usepackage{fancyhdr} % 页眉页脚

\fancyhf{} % 清除所有页眉页脚
\pagestyle{fancy}
\fancypagestyle{plain}{%改变首页
  \fancyhf{}
  \cfoot{\thepage}
  % \fancyhead[C]{大数据机器学习}
  \renewcommand{\headrulewidth}{0mm} 
}

\fancyfoot[CO, CE]{\thepage}
\renewcommand{\headrulewidth}{0mm} 
% 设置序号格式
% \renewcommand{\thesection}{\arabic{section}.}
\renewcommand{\thesection}{\arabic{section} }
% \renewcommand{\thesubsection}{\zhnum{subsection}、}
\renewcommand{\thesubsection}{\arabic{subsection}.}
\renewcommand{\thesubsubsection}{\arabic{subsubsection}).}
% 设置标题的字体
\titleformat{\section}
  {\normalfont}{\thesection}{1em}{}
\titleformat{\subsection}
  {\normalfont}{\thesubsection}{1em}{}

% 代码和伪代码支持
\RequirePackage{minted}
\RequirePackage[ruled,linesnumbered]{algorithm2e}
\usepackage{listings}
\lstset{
    basicstyle          =   \sffamily,          % 基本代码风格
    keywordstyle        =   \bfseries,          % 关键字风格
    commentstyle        =   \rmfamily\itshape,  % 注释的风格,斜体
    stringstyle         =   \ttfamily,  % 字符串风格
    flexiblecolumns,                % 别问为什么,加上这个
    % numbers             =   left,   % 行号的位置在左边
    showspaces          =   false,  % 是否显示空格,显示了有点乱,所以不现实了
    % numberstyle         =   \zihao{-5}\ttfamily,    % 行号的样式,小五号,tt等宽字体
    showstringspaces    =   false,
    % captionpos          =   t,      % 这段代码的名字所呈现的位置,t指的是top上面
    % frame               =   lrtb,   % 显示边框
}
\lstdefinestyle{Python}{
    language        =   Python, % 语言选Python
    basicstyle      =   \zihao{-5}\ttfamily,
    numberstyle     =   \zihao{-5}\ttfamily,
    keywordstyle    =   \color{blue},
    keywordstyle    =   [2] \color{teal},
    stringstyle     =   \color{magenta},
    commentstyle    =   \color{red}\ttfamily,
    breaklines      =   true,   % 自动换行,建议不要写太长的行
    columns         =   fixed,  % 如果不加这一句,字间距就不固定,很丑,必须加
    basewidth       =   0.5em,
}

% 数学支持
\RequirePackage{amssymb}
\RequirePackage{amsfonts}
\RequirePackage{amsmath}
\RequirePackage{esint}
\RequirePackage{gensymb}
\RequirePackage{mathtools}

\title{大数据机器学习第六次理论作业}
% \author{}
\begin{document}
\maketitle

\section{某公司招聘职员考察身体、业务能力、发展潜力这3项。身体分为合格1、不合格0两级,业务能力和发展潜力分为上1、中2、下3三级。分类为合格1不合格-1两级。已知10个人的数据,如下表所示。假设弱分类器为决策树桩。试用AdaBoost算法学习一个强分类器。}
\begin{figure}[H]
\centering
\includegraphics[width=0.9\linewidth]{table1.jpg}
\end{figure}
\subsection{相关知识点}
\subsubsection{弱分类器:决策树桩}
只要能够找到比随机分类效果略好的分类器,就找到了一个弱分类器。我们可以首先构建一个弱分类器决策树桩进行最基本的二分类。

决策树桩(Decision Stump)也叫做单层决策树,主要通过给定阈值(threshold)进行分类,$x<v$或$x>v$,可以看成一个根节点直接连接两个叶节点的简单决策树。
\subsubsection{AdaBoost算法学习强分类器}
我们用:
\begin{itemize}
    \item $w_{mi}$表示第i个样本第m轮分类的权重
    \item $G_m$表示第m个分类器
    \item $e_m$表示第m个分类器的错误率
    \item $\alpha_m$表示前m个分类器的权重
    \item $f_m$表示前m个分类器的线性组合。
    \item $G_x$表示最终的分类器
\end{itemize}

AdaBoost算法的输入是:
\begin{displaymath}
\begin{split}
    T=\left\{\left(x_1, y_1\right),\left(x_2, y_2\right), \cdots,\left(x_N, y_N\right)\right\},\\
    x_i \in \mathcal{X} \subseteq \mathbf{R}^n , y_i \in \mathcal{Y}=\{-1,+1\}
\end{split}
\end{displaymath}

输出是:最终的分类器$G_x$。


算法的步骤是:
\begin{itemize}
\item 初始化训练数据的起始的权值分布,每个数据的权值:$w_{1i} = \frac{1}{N},\quad i =1,2,\cdots,N$
\item  对m个弱分类器:
\begin{itemize}
    \item  得到在权值$D_m$下的训练数据集,得到弱分类器$G_m\left(x\right)$
    \item 计算$G_m$的训练误差 $e_m = \sum_{i=1}^N P\left(G_m\left(x_i\right) \neq y_i\right)=\sum_{i=1}^N w_{m i}I\left(G_m\left(x_i\right) \neq y_i\right) $
    \item 计算$G_m$的系数 $\alpha_m=\frac{1}{2} \log \frac{1-e_m}{e_m}$
    \item 更新训练数据集样本的权值$w_{m+1, i}=\frac{w_{m i}}{Z_m} \exp \left(-\alpha_m y_i G_m\left(x_i\right)\right), \quad i=1,2, \cdots, N$,其中$Z_m=\sum_{i=1}^N w_{m i} \exp \left(-\alpha_m y_i G_m\left(x_i\right)\right)$
\end{itemize}
\item 构建规范分类器的线性组合,$f(x) = \sum_{m=1}^M\alpha_mG_m(x)$,得到最终的分类器:$G(x) = sign\left(f\left(x\right)\right)$
\end{itemize}


\subsection{代码实现}
\subsubsection{弱分类器:决策树桩}
\paragraph{算法思路:}
\begin{itemize}
\item 设最小错误率
\begin{itemize}
\item 对数据集的每个特征
\begin{itemize}
    \item 对数据集的每个步长
    \begin{itemsize}
    \item 对每个不等号,建立一颗决策树并使用加权数据集进行测试,如果当前决策树错误率比最小错误率小,更新当前决策树为最佳决策树
    \end{itemize}
    \end{itemize}
\end{itemize}
\item 返回最佳决策树桩
\end{itemize}

\paragraph{代码实现单层决策树桩(参考PPT):}
\lstinputlisting[
    style       =   Python,
    % caption     =   {\bf decisionstump.py},
    label       =   {decisionstump.py}
]{decisionstump.py}

\paragraph{输出结果}
\lstinputlisting[
    style       =   Python,
    label       =   {res_1.py}
]{res_1.py}

\subsubsection{AdaBoost算法学习强分类器}
\paragraph{代码实现AdaBoost算法学习强分类器(参考PPT):}
\lstinputlisting[
    style       =   Python,
    label       =   {adaboost_1.py}
]{adaboost_1.py}
\paragraph{输出结果}
\lstinputlisting[
    style       =   Python,
    label       =   {res_2.py}
]{res_2.py}
\paragraph{结果表明:}
以决策树桩作为弱分类器,使用AdaBoost算法学习的强分类器的错误率下降。


\section{比较支持向量机、AdaBoost、逻辑斯谛回归模型的学习策略与算法。}

\begin{table}[H]
\centering
\begin{tabular}{|c|c|c|c|}
\hline
\diagbox{对比方式}{模型} & 支持向量机 & AdaBoost & 逻辑斯谛回归 \\
\hline 
特点 & \makecell{可以通过核函数解决\\非线性可分问题} & \makecell{将多个弱分类器按\\权重组合成强分类器}& \makecell{难以处理非线性分类\\问题}\\
\hline
学习策略 & 最小间隔最大化 & 加法模型逐一学习基函数 & 极大对数似然函数 \\
\hline
算法 & \makecell{SMO算法求解凸二次\\规划对偶问题  } & \makecell{前向分步算法(损失函\\数是指数损失函数)} & \makecell{梯度下降算法、\\牛顿法、拟牛顿法求\\解极大似然估计} \\   
\hline
\end{tabular} 
\end{table}

\subsubsection{相似性与不同点}
\paragraph{相似性:}
\begin{itemize}
    \item 都可用于分类。
    \item 可以把AdaBoost弱分类器看成SVM的核函数,让AdaBoost模拟最小间隔最大化的过程。SVM的目标是找到让每个支持向量距离超平面间隔不少于1的权值。AdaBoost的目标是优化分类器的权值$\alpha$,通过对弱分类器根据权重组合来构造强分类器。
    
\end{itemize}

\paragraph{不同点:}
高维数据降维后用不同模型分类的决策边界差异如图所示。


\begin{itemize}
    \item AdaBoost通过多个弱分类器的组合,让决策边界更加复杂,处理了难以分类的样本,让分类效果更好。
    \item 线性核SVM通过最小间隔最大化获得比逻辑斯谛回归的模型更优的效果。
\end{itemize}
\begin{figure}[H]
\centering
\includegraphics[width=1.1\linewidth]{comparising_adaboost_output.png}
\end{figure}
\end{document}