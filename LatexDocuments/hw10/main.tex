\documentclass{article}

% Language setting
% Replace `english' with e.g. `spanish' to change the document language
% \usepackage[english]{babel}
\usepackage[UTF8,heading = true]{ctex}

% Set page size and margins
% Replace `letterpaper' with `a4paper' for UK/EU standard size
\usepackage[a4paper,top=2cm,bottom=2cm,left=3cm,right=3cm,marginparwidth=1.75cm]{geometry}

\usepackage{float}
\usepackage{titlesec}
\usepackage{lipsum}
\usepackage{zhnumber}
% 表格
\usepackage{diagbox}
\usepackage{makecell}
% Useful packages
\usepackage{amsmath}
\usepackage{graphicx}
\usepackage[colorlinks=true, allcolors=blue]{hyperref}
\usepackage{fancyhdr} % 页眉页脚

\fancyhf{} % 清除所有页眉页脚
\pagestyle{fancy}
\fancypagestyle{plain}{%改变首页
  \fancyhf{}
  \cfoot{\thepage}
  % \fancyhead[C]{大数据机器学习}
  \renewcommand{\headrulewidth}{0mm} 
}

\fancyfoot[CO, CE]{\thepage}
\renewcommand{\headrulewidth}{0mm} 
% 设置序号格式
% \renewcommand{\thesection}{\arabic{section}.}
\renewcommand{\thesection}{\zhnum{section}、 }
% \renewcommand{\thesubsection}{\zhnum{subsection}、}
\renewcommand{\thesubsection}{\arabic{subsection}.}
\renewcommand{\thesubsubsection}{\arabic{subsubsection}).}
% 设置标题的字体
\titleformat{\section}
  {\normalfont}{\thesection}{1em}{}
\titleformat{\subsection}
  {\normalfont}{\thesubsection}{1em}{}

% 代码和伪代码支持
\RequirePackage{minted}
\RequirePackage[ruled,linesnumbered]{algorithm2e}
\usepackage{listings}
\lstset{
    basicstyle          =   \sffamily,          % 基本代码风格
    keywordstyle        =   \bfseries,          % 关键字风格
    commentstyle        =   \rmfamily\itshape,  % 注释的风格,斜体
    stringstyle         =   \ttfamily,  % 字符串风格
    flexiblecolumns,                % 别问为什么,加上这个
    % numbers             =   left,   % 行号的位置在左边
    showspaces          =   false,  % 是否显示空格,显示了有点乱,所以不现实了
    % numberstyle         =   \zihao{-5}\ttfamily,    % 行号的样式,小五号,tt等宽字体
    showstringspaces    =   false,
    % captionpos          =   t,      % 这段代码的名字所呈现的位置,t指的是top上面
    % frame               =   lrtb,   % 显示边框
}
\lstdefinestyle{Python}{
    language        =   Python, % 语言选Python
    basicstyle      =   \zihao{-5}\ttfamily,
    numberstyle     =   \zihao{-5}\ttfamily,
    keywordstyle    =   \color{blue},
    keywordstyle    =   [2] \color{teal},
    stringstyle     =   \color{magenta},
    commentstyle    =   \color{red}\ttfamily,
    breaklines      =   true,   % 自动换行,建议不要写太长的行
    columns         =   fixed,  % 如果不加这一句,字间距就不固定,很丑,必须加
    basewidth       =   0.5em,
}

% 数学支持
\RequirePackage{amssymb}
\RequirePackage{amsfonts}
\RequirePackage{amsmath}
\RequirePackage{esint}
\RequirePackage{gensymb}
\RequirePackage{mathtools}

% 编号格式
\usepackage{enumitem}

% 定义定理
\newtheorem{definition}[theorem]{Definition} 

\title{大数据机器学习第十次理论作业}
\author{常智星 2024214438}
\begin{document}
\maketitle

\section{试求矩阵$A=\left[\begin{matrix}2&4\\1&3\\0&0\\0&0\end{matrix}\right]$的奇异值分解并写出其外积展开式。}
\subsection{问题分析与相关知识点}
\subsubsection{奇异值分解}
矩阵的奇异值分解是指,将一个非零的$m\times n$实矩阵$A$,$A \in R^{m\times n}$,表示为以下三个实矩阵乘积形式的运算,即进行矩阵的因子分解:
\begin{equation}
    A = U\Sigma V^\intercal
\end{equation}
其中,$U$是$m$阶正交矩阵,$V$是$n$阶正交矩阵,$\Sigma$是由降序排列的非负对焦线元素组成的$m\times n$矩形对角矩阵,满足:
$$
U U^\intercal = I
$$
$$
V V^\intercal = I
$$
$$
\Sigma = \diag(\sigma_1,\sigma_2,\cdots,\sigma_p)
$$
$$
\sigma_1 \geq \sigma_2 \geq \cdots \geq \sigma_p \geq 0
$$
$$
p = \min(m,n)
$$
其中,$U\Sigma V^\intercal$称为矩阵的奇异值分解(singular value decomposition, SVD);$\sigma_i$称为矩阵的奇异值;$U$的列向量称为左奇异向量,$V$的列向量称为右奇异向量。
\subsubsection{矩阵的基本子空间}
\begin{enumerate}[label = {\Roman*.}]

\item 向量子空间

若S是向量空间V的非空子集,且S满足以下条件:
\begin{enumerate}[label = {\roman*.}]
\item 对任意实数$a$,若$x \in S$,则$ax \in S$;

\item 若$x \in S$且$y \in S$,则$x+y=S$;
\end{enumerate}
则S称为V的子空间。

设$v_1,v_2,\cdots,v_n$为向量空间$V$中的向量,则其线性组合
$$
a_1v_1+a_2v_2+\cdots+a_nv_n
$$
构成$V$的子空间,称$v_1,v_2,\cdots,v_n$张成的子空间,或$v_1,v_2,\cdots,v_n$的张成,记为:
$$
\text{span}(v_1,v_2,\cdots,v_n)
$$

如果$\text{span}\{v_1,v_2,\cdots,v_n\}=V$,则称称$v_1,v_2,\cdots,v_n$张成$V$。
\item 向量空间的基和维数

向量空间$V$中向量$v_1,v_2,\cdots,v_n$称为空间$V$的基,如果满足条件:
\begin{enumerate}[label = {\roman*.}]
\item $v_1,v_2,\cdots,v_n$线性无关;

\item $v_1,v_2,\cdots,v_n$张成$V$;
\end{enumerate}
向量空间基的个数即向量空间的维数。

\item 矩阵的行空间和列空间

设$A$为一个$m$行$n$列$m \times n$的矩阵,其每一行可以看作$R^n$中的一个向量,称为A的行向量;类似地,其每一列可以看作$R^m$中的一个向量,称为A的列向量。

设$A$为一个$m$行$n$列$m \times n$的矩阵,则由A的行向量张成的$R^n$的子空间,称为A的行空间;由A的列向量张成的$R^m$的子空间,称为A的列空间。

矩阵A的行空间的维数等于列空间的维数,称为矩阵的秩。

\item 矩阵的零空间

设$A$为一个$m$行$n$列$m \times n$的矩阵,令$N(A)$为齐次方程组$Ax=0$的所有解的集合,则$N(A)$为的$R^n$的一个子空间,称为$A$的零空间(null space):
\begin{equation}
    N(A) = \{x \in R^n|Ax=0\}
\end{equation}

一个矩阵零空间的维数称为矩阵的零度。

【秩-零度定理】设$A$为一$m\times n$的矩阵,则$A$的秩和$A$的零度之和为$n$。若$A$的秩为$r$,则方程组$Ax=0$的独立变量有$r$个,自由变量有$n-r$个。

\item 子空间的正交补

设$X$和$Y$为$R^n$的子空间。若对每一个$x \in X$和$y \in Y$都满足$x^\intercal y = 0$,则称$X$和$Y$是正交的,记作$X\perp Y$。

令$Y$为$R^n$的子空间,$R^n$中与$Y$中的每一个向量都正交的向量集合记为$Y^\perp$,即
$$
Y^\perp = \{x \in R|x^\intercal y =0,\forall y \in Y\}
$$
集合$Y^\perp$称为$Y$的正交补。

\item 矩阵的基本子空间

设$A$为一$m\times n$的矩阵,可以把$A$看成是将$R^n$映射到$R^m$的线性变换。一个向量$z \in R^m$在$A$的列空间的充要条件是存在$x \in R^n$,使得$z = Ax$。这样$A$的列空间和$A$的值域是相同的。记$A$的值域为$R(A)$,则:
\begin{align*}
    R(A) &= \{x \in R^m|\exists x \in R^n,z=Ax\}\\
    &=A\text{的列空间}
\end{align*}

类似地,一个向量$y \in R^n$,$y^\intercal$在$A$的行空间的充要条件是$\exists x \in R^m$,使得$y = A^\intercal x$。这样$A$的行空间和$A^\intercal$的值域$R(A^\intercal)$是相同的。
\begin{align*}
    R(A^\intercal) &= \{y \in R^n|\exists x \in R^m,y=A^\intercal x\}\\
    &=A\text{的行空间}
\end{align*}
\item 矩阵的基本子空间

矩阵有4个基本子空间:列空间、行空间、零空间、A的转置的零空间(左零空间),其关系如图1所示。
\begin{figure}[H]
    \centering
    \includegraphics[width=0.95\linewidth]{juzhenjibenzikongjian.png}
    \caption{矩阵的基本子空间的关系}
    \label{fig:my_label}
\end{figure}
有下面的定理成立:

【定理1】若$A$为一$m \times n$的矩阵,则$N(A)=R(A^\intercal)^\perp$,且$N(A^\intercal)=R(A)^\perp$。
\end{enumerate}
\subsubsection{矩阵奇异值分解的计算}
\begin{enumerate}[label = {\Roman*.}]
    \item 求$A^TA$的特征值和特征向量:$$
    W = A^\intercal A
    $$
    $$
    (W-\lambda I)x=0
    $$
    得到特征值$\lambda_i$,并将特征值从大到小排列:
    $$
    \lambda_1\geq\lambda_2\geq\cdots\lambda_n\geq0
    $$
    并将特征值$\lambda_i(i=1,2,\cdots,n)$代入特征方程求得对应的特征向量。
    \item 求$n$阶正交矩阵$V$:将特征向量单位化得到单位特征向量$v_1,v_2,\cdots,v_n$,构成n阶正交矩阵$V$:
    $$
    V = \left[\begin{matrix}v_1&v_2&\cdots&v_n\end{matrix}\right]
    $$
    \item 求$m\times n$对角矩阵$\Sigma$:计算$A$的奇异值:
    $$
    \sigma_i = \sqrt{\lambda_i},\quad i =1,2,\cdots,n
    $$
    构造$m\times n$对角矩,主对角线元素是奇异值,其余元素为0:
    $$
    \Sigma = \text{diag}(\sigma_1,\sigma_2,\cdots,\sigma_n)
    $$
    \item 求$m$阶正交矩阵$U$:对$A$的前$r$个正奇异值,令:
    $$
    u_j = \frac{1}{\sigma_j}Av_j,\quad j = 1,2,\cdots,r
    $$
    得到
    $$
    U_1 = \left[\begin{matrix}u_1&u_2&\cdots&u_r\end{matrix}\right]
    $$
    求$A^T$的零空间的一组标准正交基$\{u_{r+1},u_{r+2},\cdots,u_m\}$,令
    $$
    U_2 = \left[\begin{matrix}u_{r+1}&u_{r+2}&\cdots&u__m\end{matrix}\right]
    $$
    并令
    $$
    U = \left[\begin{matrix}U_1&U_2\end{matrix}\right]
    $$
    \item 得到奇异值分解:
    $$
    A = U\Sigma V^\intercal
    $$
    
    
\end{enumerate}
\subsubsection{矩阵的外积展开式}
矩阵的奇异值分解$U\Sigma V^\intercal$也可以由外积形式表示。
如果把A的奇异值分解,看成矩阵$U\Sigma$和$V^\intercal$的乘积,将$U\Sigma$按列向量分块,将$V^\intercal$按行向量分块,即得到:
$$
U\Sigma = \left[\begin{matrix} \sigma_1u_1 & \sigma_2u_2 & \cdots & \sigma_nu_n \end{matrix}\right]
$$
$$
V^\intercal = \left[\begin{matrix}v_1^\intercal \\ v_2^\intercal \\ \vdots \\ v_n^\intercal\end{matrix}\right]
$$
则$A$的外积展开式:
$$
A = \sigma_1u_1v_1^\intercal+\sigma_2u_2v_2^\intercal+\sigma_n u_n v_n^\intercal
$$
即:
$$
u_iv_j^\intercal = \left[\begin{matrix}u_{1i} \\ u_{2i} \\ \vdots \\ u_{mi}\end{matrix}\right] \left[\begin{matrix} v_{1j} & v_{2j} & \cdots & v_{nj} \end{matrix}\right]
 = \left[\begin{matrix}u_{1i}v_{1j} & u_{1i}v_{2j} & \cdots & u_{1i}v_{nj}\\u_{2i}v_{1j} & u_{2i}v_{2j} & \cdots & u_{2i}v_{nj}\\\vdots&\vdots&\vdots&\vdots\\u_{mi}v_{1j} & u_{mi}v_{2j} & \cdots & u_{mi}v_{nj}\end{matrix}\right]
$$
$A$的外积展开式也可以写为:
$$
A = \sum_{k=1}^n A_k = \sum_{k=1}^n \sigma_k u_k v_k^\intercal
$$
由$A$的外积展开式可知,若$A$的秩为$n$,则:
$$
A = \sigma_1u_1v_1^\intercal+\sigma_2u_2v_2^\intercal+\sigma_nu_nv_n^\intercal
$$
设矩阵
$$
A_k = \sigma_1u_1v_1^\intercal+\sigma_2u_2v_2^\intercal+\sigma_k u_k v_k^\intercal
$$
则$A_k$的秩为$k$,并且$A_k$是秩为$k$的矩阵在弗罗贝尼乌斯范数意义$A$的最优近似矩阵,矩阵$A_k$就是$A$的截断奇异值分解。
由于通常奇异值$\sigma_i$递减很快,所以$k$取很小时,$A_k$也能对$A$有很好的近似。

\subsection{问题求解}
\subsubsection{求矩阵的奇异值分解}
\begin{enumerate}[label = {\Roman*.}]
    \item 求$A^\intercal A$的特征值和特征向量:$$
    A^\intercal A = \left[\begin{matrix}2&1&0\\4&3&0\end{matrix}\right] \left[\begin{matrix}2&4\\1&3\\0&0\\0&0\end{matrix}\right] = \left[\begin{matrix}5&11\\11&25\end{matrix}\right]
    $$
    特征值和特征向量满足方程:
    $$
    (A^\intercal A-\lambda I)x=0
    $$
    得到其次线性方程组:
    $$
    \begin{cases}
    (5-\lambda)x_1 + 11&x_2 = 0\\
    11x_1 + (25-\lambda)&x_2 = 0
    \end{cases}
    $$
    方程组有解的充要条件是行列式
    $\begin{vmatrix}5-\lambda&11\\11&25-\lambda\end{vmatrix}=0$
    
    即
    $$
    (5-\lambda)(25-\lambda)-121=\lambda^2-30\lambda+4=0
    $$
    可求解得到特征值$\lambda_i$,并将特征值从大到小排列:
    $$
    \lambda_1 = 15+\sqrt{221},\quad \lambda_2 = 15-\sqrt{221}
    $$
    为方便计算,使用代码进行后续求解。
    \item 代码实现
\lstinputlisting[
    style       =   Python,
    label       =   {svd_1.py}
]{svd_1.py}
\paragraph{结果}
\lstinputlisting[
    style       =   Python,
    label       =   {svd_1_res.py}
]{svd_1_res.py}
\end{enumerate}
\subsubsection{外积展开式表示}
根据上面的计算结果可知:
$$
\sigma_1 = 5.4650,\quad \sigma_2 = 0.3660
$$
$$
v_1^\intercal = \left[\begin{matrix}0.4046& -0.9145\end{matrix}\right],\quad v_2^T = \left[\begin{matrix}-0.9145&0.4046\end{matrix}\right]
$$
$$
u_1 = \left[\begin{matrix}-0.5213\\-0.4280\end{matrix}\right],\quad u_2 = \left[\begin{matrix}-9.4196\\-5.8152\end{matrix}\right]
$$

A的外积展开式:
$$
A = \sigma_1 u_1 v_1^\intercal +\sigma_2 u_2 v_2^\intercal
$$

\paragraph{代码实现}
\lstinputlisting[
    style       =   Python,
    label       =   {svd_1_outer.py}
]{svd_1_outer.py}
\paragraph{结果}
\lstinputlisting[
    style       =   Python,
    label       =   {svd_1_outer_res.py}
]{svd_1_outer_res.py}

\newpage
\section{搜索中的点击数据记录用户搜索时提交的查询语句,点击的网页URL以及点击的次数构成一个二部图,其中一个结点集合$\{q_i\}$表示查询,另一个结点集合$\{u_j\}$表示URL,边表示点击关系,边上的权重表示点击次数。图15.2是一个简化的点击数据例。点击数据可以由矩阵表示,试对该矩阵进行奇异值分解,并解释得到的三个矩阵所表示的内容}
\begin{figure}[H]
    \centering
    \includegraphics[width=0.3\linewidth]{fig-15-2.png}
    \label{fig:my_label}
\end{figure}
\subsection{问题求解}
\subsubsection{写出二部图对应的矩阵A}
$$\quad\quad\quad \begin{matrix}u_1&u_2&u_3&u_4&u_5\end{matrix}
$$
$$
A =\begin{matrix}q_1\\q_2\\q_3\\q_4\end{matrix} \left[\begin{matrix}0&20&5&0&0\\10&0&0&3&0\\0&0&0&0&1\\0&0&1&0&0\end{matrix}\right]
$$
\subsubsection{求矩阵A的奇异值分解:}
\lstinputlisting[
    style       =   Python,
    label       =   {svd_2_point.py}
]{svd_2_point.py}
\paragraph{结果}
\lstinputlisting[
    style       =   Python,
    label       =   {svd_2_point_res.py}
]{svd_2_point_res.py}
\subsubsection{分析3个矩阵代表的含义}
\begin{enumerate}[label = {\Roman*.}]
\item 矩阵$\Sigma$:表示每个网页特征的重要程度。
\item 矩阵$V$:$V$的列向量表示每个URL与每个网页特征之间的关系。

矩阵$V$的第1列表示URL1的第2个特征较显著;第2列表示URL2的第1个特征较显著。

\item 矩阵$U$:$U$的列向量表示每个查询和每个网页特征之间的对应关系。

矩阵$U$的第1列表示第1个查询倾向于第1个特征较显著的URL,即URL2;矩阵U的第2列表示第2个查询倾向于第2个特征较显著的URL,即URL1。

\end{enumerate}
\end{document}